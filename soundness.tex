
A type soundness theorem relates the type $\stype_0$ of a well-formed
 expression $\sexpr_0$ to the possible outcomes of evaluation.\footnote{The
  definition considers closed $\sexpr_0$ without loss of generality;
  to study an open expression, wrap it in a closing context or lambda.
  By contrast, prior works that employ an ``open-world'' soundness theorem
  restrict the syntax of mixed-typed programs~\cite{tf-dls-2006, vss-popl-2017}.}
In particular, if the evaluation of $\sexpr_0$ results in
 a value $\svalue_0$, then the surface type $\stype_0$ predicts some properties
 of $\svalue_0$ in the evaluation language.
For each combination of surface typing~($\sWT$) and evaluation typing
  ($\sWT_X$ for $\scriptstyle X \in \eset{\nsym, \tsym, \asym}$) judgments,
  the predictive aspect of a type soundness theorem may be expressed as a
  function $\sXproj$ on types.

% TODO
% - transient needs the heap to establish the type of the value
\begin{definition}[type soundness]
  Let $\sXproj$ be a function from surface types to evaluation types.
  % NOTE: F maybe has type `\vdash -> \vdash_X`
  A reduction relation $\rredX$ satisfies $\propts{\sWTX}{\sXproj}$ iff for all
  $\fwellformed{\sexpr_0}{\stype_0}$ one of the following holds:
  \begin{itemize}
    \itemsep0.5ex
    \item \label{clause:F}
      $\sexpr_0 \rredX \svalue_0$
      and $\sWTX \svalue_0 : \sXproj(\stype_0)$
    \item
      $\sexpr_0 \rredX \sexpr_1$
      and $\sexpr_1 \in \eset{\tagerrorD{}, \divisionbyzeroerror} \cup \boundaryerror{\sbset}{\svalue}$
    \item
      $\fdiverge{\sexpr_0}{\rredX}$
  \end{itemize}
\end{definition}

By implication, type soundness states evaluation does not reach certain
 ``wrong'' states~\cite{m-jcss-1978}: static tag errors ($\tagerrorS{}$)
 and irreducible expressions.

To formulate the type soundness theorem for the three semantics, we
require two functions on types.
The first, $\stagproj$ from \figref{fig:evaluation-metafunctions}, maps a surface type
to its constructor.
The second, $\sidproj$, is the identity function on types.

\begin{theorem}[type soundness]\label{thm:type-soundness}
 \leavevmode
  \begin{enumerate}
    \itemsep0.5ex
    \item $\rredN$ satisfies $\propts{\sWTA}{\sidproj}$
    \item $\rredT$ satisfies $\propts{\sWTT}{\stagproj}$
    \item $\rredA$ satisfies $\propts{\sWTA}{\sidproj}$
  \end{enumerate}
\end{theorem}
\begin{proofsketch}
  By three lemmas per semantics: progress, preservation, and that surface
  typing implies evaluation typing.  The key for the first bullet is that
  the evaluation syntax of \Aname{} extends the one of \Nname{} and the
  type judgment $\sWTA$ extends $\sWTN$; hence the proof can verify the
  theorem for the {\em same\/} type judgment.  See the
  \techreport{} for full proofs.
\end{proofsketch}

% \subsection{Type soundness fails to account for meaningful distinctions}

