
Ownership labels decorate an expression with the names of the currently-responsible
components. In
the original program, each component owns all of its subexpressions. An
expression reduces to a value; if a value crosses a boundary, it gains
a label.
A value loses a label only when it
fully matches a boundary type.  This section provides a formal language of
ownership and explains:
\begin{itemize}
  \item how to lift an expression into an ownership-labeled syntax, and
  \item how to lift a reduction relation to propagate ownership labels in a path-based sense.
\end{itemize}

\subsection{Ownership-labeled Syntax}

\begin{figure}\flushleft
  \begin{minipage}[t]{0.35\columnwidth}\vspace{-8mm}
  \lbl{\fbox{Ownership Language}}{
    \begin{langarray}
      \sexpr & \BNFeq &
        \ldots
        \mid \obars{\sexpr}{\sowner}
        \mid \edyn{\sbnd}{\obars{\sexpr}{\sowner}} \mid
      \\ & &
        \esta{\sbnd}{\obars{\sexpr}{\sowner}}
        \mid \emon{\sbnd}{\obars{\sexpr}{\sowner}}
      \\
      \svalue & \BNFeq &
        \ldots
        \mid \obars{\svalue}{\sowner}
      \\
      \sowner & \BNFeq &
        \textrm{\scountable{} set} %, with an implicit mapping to component names}
        % TODO what to call them? boundary labels, component names, blame labels, 
        % make the blame/ownership distinction clear in explanation, the new type is only superscripts
      \\
      \sowners & \BNFeq &
        \powerset{\sowner}
      \\
      \sownerlist & \BNFeq &
        \ % empty ... use ldots instead?
        \mid \fconcat{\sowner}{\sownerlist}
    \end{langarray}
  }

  \end{minipage}\begin{minipage}[t]{0.65\columnwidth}
    \(\begin{array}[t]{l@{~~}c@{~}l}
    \fvalueowners{\svalue_0} & \feq &
      \left\{\begin{array}{ll}
        \eset{\sowner_0} \cup \fvalueowners{\svalue_1}
         & \mbox{if $\svalue_0 \eeq \obars{\svalue_1}{\sowner_0}$}
        \\
        \fvalueowners{\svalue_1}
         & \mbox{if $\svalue_0 \eeq \ehist{\sbset_0}{\svalue_1}$}
        \\
        \eset{}
         & \mbox{otherwise}
      \end{array}\right.
    \\\\
    \frev{\sowner_0 \ldots \sowner_n} & \feq &
      \sowner_n \ldots \sowner_0
    \\\\
      \zerowidth{
        \sexpr_0 \eeq \obbars{\sexpr_1}{\sownerlist_0}
        \sabbreveq
        \sexpr_0 \eeq \obars{\cdots \obars{\sexpr_1}{\sowner_n} \cdots}{\sowner_1}}
  \end{array}\)

  \end{minipage}

  \medskip
  \begin{minipage}[t]{0.46\columnwidth}
  \lbl{\fbox{$\fwellformedO{\sexpr}{\stoptional}$}}{\(\begin{array}{l@{}l} % well-formed expression
    \fwellformedO{\obars{\sexpr_0}{\sowner_0}}{\stype_0} &
    \mbox{ iff } \sowner_0 \sWSOP \obars{\sexpr_0}{\sowner_0}
    \mbox{ and} \sWTO \obars{\sexpr_0}{\sowner_0} : \stype_0\!\!
    \\
    \fwellformedO{\obars{\sexpr_0}{\sowner_0}}{\tdyn} &
    \mbox{ iff } \sowner_0 \sWSOP \obars{\sexpr_0}{\sowner_0}
    \mbox{ and} \sWTO \obars{\sexpr_0}{\sowner_0} : \tdyn\!\!
  \end{array}\)}{}

  \end{minipage}\begin{minipage}[t]{0.264\columnwidth}
    \lbl{\fbox{$\stypeenv\!\sWTO\!\sexpr\!:\!\stype\!$} \missingrules{}}{\begin{mathpar}
      \inferrule*{
        \stypeenv_0 \sWTO \sexpr_0 : \stype_0
      }{
        \stypeenv_0 \sWTO \obars{\sexpr_0}{\sowner_0} : \stype_0
      }
    \end{mathpar}}

  \end{minipage}\begin{minipage}[t]{0.264\columnwidth}
    \lbl{\fbox{$\stypeenv\!\sWTO\!\sexpr : \tdyn\!$} \missingrules{}}{\begin{mathpar}
      \inferrule*{
        \stypeenv_0 \sWTO \sexpr_0 : \tdyn
      }{
        \stypeenv_0 \sWTO \obars{\sexpr_0}{\sowner_0} : \tdyn
      }
    \end{mathpar}}
  \end{minipage}

  \smallskip
  \lbl{\fbox{$\sownerenv; \sowner \sWSOP \sexpr$} \missingrules{}}{\begin{mathpar} %  well-named-and-labeled components (selected rules)
    \inferrule*{
      \sownerenv_0; \sowner_0 \sWSOP \sexpr_0
    }{
      \sownerenv_0; \sowner_0 \sWSOP \obars{\sexpr_0}{\sowner_0}
    }

    \inferrule*{
      \tann{\svar_0}{\sowner_0} \in \sownerenv_0
    }{
      \sownerenv_0; \sowner_0 \sWSOP \svar_0
    }

    \inferrule*{
      \fcons{\tann{\svar_0}{\sowner_0}}{\sownerenv_0}; \sowner_0 \sWSOP \sexpr_0
    }{
      \sownerenv_0; \sowner_0 \sWSOP \efun{\svar_0}{\sexpr_0}
    }

    \inferrule*{
      \fcons{\tann{\svar_0}{\sowner_0}}{\sownerenv_0}; \sowner_0 \sWSOP \sexpr_0
    }{
      \sownerenv_0; \sowner_0 \sWSOP \efun{\tann{\svar_0}{\stype_0}}{\sexpr_0}
    }

    \inferrule*{
      \sownerenv_0; \sowner_0 \sWSOP \sexpr_0
      \quad
      \sownerenv_0; \sowner_0 \sWSOP \sexpr_1
    }{
      \sownerenv_0; \sowner_0 \sWSOP \eapp{\stoptional}{\sexpr_0}{\sexpr_1}
      %\sownerenv_0; \sowner_0 \sWSOP \ebinop{\sexpr_0}{\sexpr_1}
    }

    \inferrule*{
      \sownerenv_0; \sowner_1 \sWSOP \sexpr_0
    }{
      \sownerenv_0; \sowner_0 \sWSOP \edynb{\obnd{\sowner_0}{\stype_0}{\sowner_1}}{\obars{\sexpr_0}{\sowner_1}}
    }

    \inferrule*{
      \sownerenv_0; \sowner_1 \sWSOP \sexpr_0
    }{
      \sownerenv_0; \sowner_0 \sWSOP \estab{\obnd{\sowner_0}{\stype_0}{\sowner_1}}{\obars{\sexpr_0}{\sowner_1}}
    }

    \inferrule*{
      \sownerenv_0; \sowner_1 \sWSOP \svalue_0
    }{
      \sownerenv_0; \sowner_0 \sWSOP \emon{\obnd{\sowner_0}{\stype_0}{\sowner_1}}{\obars{\svalue_0}{\sowner_1}}
    }

    \inferrule*{
      \sownerenv_0; \sowner_0 \sWSOP \svalue_0
    }{
      \sownerenv_0; \sowner_0 \sWSOP \ehist{\sbset_0}{\svalue_0}
    }

  \end{mathpar}}

  \Description[Grammar extensions, metafunctions, and judgment forms.]{Syntax
    and judgments for ownership. Expressions and values may have labels;
    boundary expressions must have labels. An `owners' metafunction gets
    the labels from a value; a `rev' metafunction reverses a sequence of labels.
    An abbreviation captures a sequence of labels. There is a new well-formedness
    judgment, which relies on extended typing judgments and an extended
    labeling judgment. The typing judgments accept labeled expressions. The
    labeling judgment checks that every boundary has a label.
  }
  \caption{Ownership language extensions}
  \label{fig:annotated-syntax}
\end{figure}

\noindent The addition of ownership labels adds new syntax and one
requirement to an evaluation language.
Ownership labels $\sowner$ form a new syntactic category, but correspond to
component names.
Labels annotate expressions as superscripts:
$\obars{\sexpr}{\sowner}$.
The new requirement is that in each boundary term the enclosed expression
comes with a label that matches the sender name, e.g.,
the unlabeled term $(\estab{\obnd{\sowner_0}{\stype_0}{\sowner_1}}{\sexpr_0})$
must add the label $\sowner_1$ to the inner expression.

Initially, only the immediate subexpressions of boundary terms have an explicit label.
All other expressions implicitly have the same owner as the context they appear
in.
Contracting a redex may combine labels from different
subexpressions onto the same term. Hence expressions and values may become
nested under a sequence of labels.  A value that occurs under
several distinct ownership labels (including the label of its context) has
multiple owners.

\Figref{fig:annotated-syntax} defines an extended grammar, metafunctions,
 and judgments for an ownership-labeled language.
The main judgment, $\fwellformedO{\sexpr}{\stoptional}$,
 states that a well-formed expression has a top-most owner,
 contains names and labels that satisfy the $\sWSOP$ judgment,
 and is well-typed.
The $\sWSOP$ judgment formalizes an \emph{ownership consistency relation}\/
 with respect to a name $\sowner$ for the enclosing component and a map
 $\sownerenv$ from variables to component names.\footnote{\Tname{} additionally
 requires a heap labeling. The details are standard and appear in the
 \techreport{}.}
 Within one component, consistency
 requires that every labeled expression $\obars{\sexpr_0}{\sowner_0}$
 matches the enclosing component and that every variable is bound by a
 function defined in a matching component.  Other values and non-boundary
 expressions satisfy consistency if their subexpressions do.  At the
 boundaries between components, the client side of the boundary
 specification must match the enclosing component and the sender-side name
 must match the required label on the subexpression.

The metafunction $\svalueowners$\/ formally defines the owners of a value
 as the set of labels around an unlabeled value stripped of any trace-wrapper
 metadata.
Monitor wrappers are not stripped because they represent boundaries.
\Figref{fig:annotated-syntax} lastly overloads the $\srev$\/ metafunction
 to reverse a sequence of labels, and defines the abbreviation
 $\obbars{\sexpr}{\sownerlist}$ to capture labels around an expression.
For example, $\obbars{4}{\fconcat{\sowner_0}{\sowner_1}}$
 is short for $\obars{\obars{4}{\sowner_0}}{\sowner_1}$
 and $\obbars{5}{\sownerlist_0}$ matches $5$ with $\sownerlist_0$ bound to
 the empty sequence.


%% -----------------------------------------------------------------------------
\subsection{Ownership-labeled Reduction}

A notion of reduction for an evaluation language may be systematically lifted
 to a labeled language in two steps.
Each lifted rule must first handle terms with an arbitrary number of
 ownership labels per expression
 and must second propagate labels to reflect changes in the expression.

The following laws explain how to propagate labels.
Each handles one scenario in which labels may be
 transferred or dropped and comes with an example reduction rule.
No new labels may be created during a reduction:

{\begin{enumerate}
    \itemsep1ex
    % creation = (+ 2 2) creates 4, dyn Nan <1,2> creates <...,...>
    %[law of creation]
    \item \label{ax:new}
      Every newly-created value is owned by the enclosing context.
    \subitem
      $\obars{\esum{\tnat}{\obars{2}{\sowner_0}}{\obars{2}{\sowner_1}}}{\sowner_2}
       \rrarrow \obars{4}{\sowner_2}$
    \subitem
      \emph{The result\/ $4$ is a new value introduced by the runtime system via\/ the $\sdelta$ metafunction.}

    % positive elim = fst/snd projection, result of fun. application, ...
    %[law of positive elimination]
    \item \label{ax:pos}
      Every value that flows out of a value $\svalue_0$
      acquires the labels of $\svalue_0$ and the context.
    \subitem
      $\obars{\esnd{\tdyn}{\obbars{\epair{\obars{1}{\sowner_0}}{\obars{2}{\sowner_0}}}{\fconcat{\sowner_1}{\sowner_2}}}}{\sowner_3}
       \rrarrow \obbars{2}{\fconcat{\sowner_0}{\fconcat{\sowner_1}{\fconcat{\sowner_2}{\sowner_3}}}}$
    \subitem
      \emph{The value\/ $2$ flows out of the pair\/ $\epair{1}{2}$ and thereby acquires the labels on the pair.}

    % negative elim = argument to function, ...
    %[law of negative elimination]
    \item \label{ax:neg}
      Every value that flows into $\svalue_0$ acquires the label
      of the context and the reversed labels of $\svalue_0$.
    \subitem
      $\newcommand{\thevalue}{\epair{8}{6}}
       \obars{\eapp{\tdyn}{\obbars{\efun{\svar_0}{\efst{\tdyn}{\svar_0}}}{\fconcat{\sowner_0}{\sowner_1}}}{\obars{\thevalue}{\sowner_2}}}{\sowner_3}
       \rrarrow
       \obars{\obbars{\efst{\tdyn}{\obbars{\thevalue}{\fconcat{\sowner_2}{\fconcat{\sowner_3}{\fconcat{\sowner_1}{\sowner_0}}}}}}{\fconcat{\sowner_0}{\sowner_1}}}{\sowner_3}$
    \subitem
      \emph{The argument value\/ $\epair{8}{6}$ is input to the function. The substituted body\/ $(\efst{\tdyn}{\epair{8}{6}})$}
    \subitem
      \emph{flows out of the function, and by \lawref{ax:pos} acquires the function's labels.}

    %[law of base checks]
    \item \label{ax:base}
      If a base value reaches a boundary with a matching base type,
      then the value may drop its current labels and
      cross the boundary as a new value in the new context.
    \subitem $\newcommand{\thevalue}{0}
              \obars{\estab{\obnd{\sowner_0}{\tnat}{\sowner_1}}{\obars{\thevalue}{\fconcat{\sowner_2}{\sowner_1}}}}{\sowner_0}
              \rrarrow \obars{\thevalue}{\sowner_0}$
    \subitem
      \emph{The value\/ $0$ fully matches the type\/ $\tnat$\/ $(\ftagof{\tnat} \eeq \knat \wedge \fshallow{\ftagof{\tnat}}{0})$.}

    %[law of no-check transfer]
    \item \label{ax:cross}
      Any other value that crosses a boundary must acquire the label of
      the new context.
    \subitem
      $\newcommand{\thevalue}{\epair{{-2}}{1}}
                \obars{\estab{\obnd{\sowner_0}{\tnat}{\sowner_1}}{\obars{\thevalue}{\sowner_1}}}{\sowner_0}
                \rrarrow \obbars{\thevalue}{\fconcat{\sowner_1}{\sowner_0}}$
    \subitem
      \emph{The pair\/ $\epair{{-2}}{1}$ does not match the type $\tnat$ and therefore keeps its old label.}

    \item \label{ax:dup}
      Consecutive equal labels may be dropped.
      \subitem $\obbars{0}{\fconcat{\sowner_0}{\fconcat{\sowner_0}{\fconcat{\sowner_1}{\sowner_0}}}} \eeq \obars{0}{\fconcat{\sowner_0}{\fconcat{\sowner_1}{\sowner_0}}}$

    \item \label{ax:error}
      Labels on an error term may be dropped.
      \subitem $\ctx[\obars{\divisionbyzeroerror}{\sowner_0}] \rrarrow \divisionbyzeroerror$

  \end{enumerate}}

 All told, the laws reflect an algebraic intuition about how
 values travel across a program.  The same intuition is implicit in
 \secref{sec:motivation} and further motivated in prior work on higher-order
 contracts~\cite{dtf-esop-2012}.
Note that function application is the only ``input'' operation in our models;
 the addition of mutable cells would add another kind of input.

% \footnote{Any set of LAWS is subject to criticism and competition.
%  \Secref{sec:tradeoff} outlines a heap-based form of ownership tailored to
%  \Tname{}, though it remains to be seen whether this neutered alternative
%  is compelling.}

Lifting does not otherwise change a reduction relation, meaning the
behavior of any expression remains the same. After all, the purpose of
lifting a base semantics to a labeled language is to prove properties (that
can be stated in terms of ownership labels) about the base semantics.

\subsection{Lifting by Example}

As a first application of the laws, let us modify the \Nname{} rule that
 creates a new monitor for an untyped function that reaches a boundary.
The unlabeled rule follows:

\smallskip\(\begin{array}{l}
  \edynb{\obnd{\sowner_0}{(\tfun{\stype_0}{\stype_1})}{\sowner_1}}{\svalue_0}
    \nredNS
    \emon{\obnd{\sowner_0}{(\tfun{\stype_0}{\stype_1})}{\sowner_1}}{\svalue_0}
  \\\mbox{\quad if $\fshallow{\ftagof{\tfun{\stype_0}{\stype_1}}}{\svalue_0}$}
\end{array}\)\smallskip

Lifting the left-hand side of the rule introduces
 a label $\sowner_3$ for the owner of the context and a sequence of owners
 $\sownerlist_2$ around the untyped function:

\smallskip\(\begin{array}{l}
  \obars{\edynb{\obnd{\sowner_0}{(\tfun{\stype_0}{\stype_1})}{\sowner_1}}{\obbars{\svalue_0}{\sownerlist_2}}}{\sowner_3}
    \nredNSanns
    ~~\ldots
  \\\mbox{\quad if $\fshallow{\ftagof{\tfun{\stype_0}{\stype_1}}}{\svalue_0}$}
\end{array}\)\smallskip

\noindent
The lifted rule does not require that the names in the boundary specification
 match the ownership labels.
In particular, both $\sowner_0 \neq \sowner_3$ and $\sowner_1 \neq \sownerlist_2$
 may hold of a labeled redex.

The propagation laws assign the label of the context to the newly-created
 monitor (\lawref{ax:new}):

\(\begin{array}{l}
  \obars{\edynb{\obnd{\sowner_0}{(\tfun{\stype_0}{\stype_1})}{\sowner_1}}{\obbars{\svalue_0}{\sownerlist_2}}}{\sowner_3}
    \nredNSanns
    \obars{\emon{\obnd{\sowner_0}{(\tfun{\stype_0}{\stype_1})}{\sowner_1}}{\obbars{\svalue_0}{\sownerlist_2}}}{\sowner_3}
  \\\mbox{\quad if $\fshallow{\ftagof{\tfun{\stype_0}{\stype_1}}}{\svalue_0}$}
\end{array}\)\smallskip

\noindent
No other laws apply.
Laws~\ref{ax:pos} and \ref{ax:neg} do not apply because only one value is
 involved and nothing flows out of it.
Laws~\ref{ax:base} and \ref{ax:cross} do not apply because $\svalue_0$ does
 not cross a boundary; rather, the monitor preserves the $\sowner_0$--$\,\sowner_1$
 boundary.
\Lawref{ax:dup} does not apply because the rule does not create or extend
 a sequence of labels, and \lawref{ax:error} does not apply because there are
 no errors.

For a second example, the following lifted \Aname{} rule projects the
 first element of a traced and monitored pair in a dynamically-typed context.
The left side of the rule introduces
 one list of owners around the trace wrapper ($\sownerlist_4$),
 another list around the monitor ($\sownerlist_3$),
 one label for the context within the monitor ($\sowner_2$),
 and one label for the outer context ($\sowner_5$).
The rule creates a $\sfst$ application inside the monitor boundary:

\vspace{-2mm}
\(\begin{array}{l}
  \obars{\efst{\tdyn}{\obbars{\ehopt{\sbset_0}{\obbars{\emon{\obnd{\sowner_0}{(\tpair{\stype_0}{\stype_1})}{\sowner_1}}{\obars{\svalue_0}{\sowner_2}}}{\sownerlist_3}}}{\sownerlist_4}}}{\sowner_5}
  \\\hspace{0.42\columnwidth}
  \nredADanns
  \obars{\eprehist{\sbset_0}{\obbars{\estab{\obnd{\sowner_0}{\stype_0}{\sowner_1}}{\obars{\efst{\stype_1}{\svalue_0}}{\sowner_2}}}{\fconcat{\sownerlist_3}{\sownerlist_4}}}}{\sowner_5}
\end{array}\)\smallskip

\noindent
The new $\ssta$ boundary mediates between the same two components as the monitor,
 and therefore has the same $\sowner_2$ label.
\Lawref{ax:new} applies this label to the $\sfst$ application.
\Lawref{ax:pos} determines the labels outside the $\ssta$ expression
 because this expression is the result of eliminating two wrappers.
The other laws do not apply.

Lifted variants of the semantics from \secref{sec:semantics} appear
 in the \techreport{}.
Henceforth, the symbol $\rredXanns\,$ refers to the lifted variant
 of the $\rredX$ reduction relation.

