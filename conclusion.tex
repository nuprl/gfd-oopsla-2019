\citeN{gf-icfp-2018} demonstrated that Typed Racket and Transient
Reticulated Python satisfy different type soundness theorems. This paper
shows that these languages also differ in how they protect
untyped code from faulty type specifications. To explain this difference,
we adapt the notion of complete monitoring from contracts to gradual
typing.

The adaptation of complete monitoring also provides a tool for explaining
the differences in blame between the semantics. While the \Nname{} semantics
points to a single boundary between typed and untyped code, \Tname{} delivers a
set of boundaries that may point to irrelevant ones and may miss crucial ones,
relative to a standard path-based notion of ownership.

The creation of \Aname{} illustrates the use of complete monitoring as
a design tool. The \Aname{} semantics satisfies the same type soundness
as \Nname{} and performs the same first-order checks
as \Tname{}. \Aname{} does not satisfy complete monitoring, but its
blame-assignment policy satisfies both soundness and completeness.
Although we do not consider \Aname{} ready for
implementation, its very existence validates the power of complete
monitoring as a design guideline.

In sum, we demonstrate that complete monitoring is an effective tool for
researchers working on mixed-typed languages. It helps with both the
analysis and design of the semantics of typing systems. Adding complete
monitoring to the researchers' tool box will greatly improve future
explorations of this area. 


\section*{Technical Report}

The \techreport{} is available from the Northeastern University Library Digital
Repository Service (\href{"http://hdl.handle.net/2047/D20202641"}{DRS})~\cite{gfd-tr-2019}.
